
\chapter{Introduction}

In the realm of sports, namely professional soccer, the use of statistical analysis has become a nearly indispensable tool for understanding the game’s complexities of match success. The conclusions that are derived from these analytics bear a very large impact on how team’s approach their game day strategy, along with the players they sign, the changes they make to their system, and how they aim to capitalize on their individual strengths. However, the transition towards a more quantitative approach to the game only came late in the 20th century with the advent of computer technology, as teams and leagues began keeping track of the game's simple metrics. As resources grew available, clubs began investing more resources into analyzing complex data. These analytics grew more sophisticated - clubs didn't just want to know what happened, but why it happened and how they could use that knowledge to predict future performance.

In soccer, match performance can be defined as the interaction of different technical, tactical, mental and physiological factors \cite{Sarmento2014}. Although specific to the sport, there exists a long list of factors which can be identified from match play, and are being leveraged today with the intent of analyzing match performance and predicting future results. Match analysis, specifically, is not very well covered in terms of systematic reviews in the sport, so the purpose of this study is therefore to systematically review and organize a portion of the literature on match analysis in adult male football as an attempt to identify the most common research topics, characterize their methodologies, and identify key characteristics and differences across unique platforms of the sport. The sport's vast historical data and global presence grants us the ability to analyze a plethora of tactical, contextual, and gameplay-related variables, and better explore how there still remain challenges in perfecting the predictive models used to determine success. 

Existing systematic reviews such as  Saremento, 2014 \cite{Sarmento2014} will be used to share results, analyze their approach to analysis, as well as to provide a basis of comparison and identify any consistencies or dissonances in findings. Other research in the field will be referenced often to share findings that answer the fundamental question at hand and illustrate how research, methods of analysis, and data-driven conclusions can vary significantly. 

Moving forward, this study will organize existing research, identify levels of KPIs in the sport, quantitatively assess the effectiveness of chosen data points, analyze gaps in this research and  discuss the implications of the results. 
