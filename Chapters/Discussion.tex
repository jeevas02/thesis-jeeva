\chapter{Discussion}

Provided here is a compiled list of notable statistically significant variables that this thesis identified, along with their specific context and again referencing the study itself for further investigation.

\subsubsection{Descriptive Variables}

Sprinting Activities: Andrzejewski et al. (2013) \cite{Andrzejewski2013} reported forwards cover the longest sprint distances and partake in short-distance sprints most often, analyzing professional soccer players' sprinting activities during the Europa League matches in the 2008 - 09 and 2010 - 11 seasons. The study underscores the importance of position-specific training regimes that cater to the unique demands of each player's role.

\subsubsection{Comparative Variables}
Physical and Technical Performance: Rampinini et al. (2009) \cite{Rampinini2009} indicated successful teams in Italy’s Serie A league demonstrated higher technical involvements than less successful teams.

High-intensity Running: Bradley et al. (2010) \cite{Bradley2010} found no significant differences in high-intensity running between elite domestic and international players, suggesting uniform physical demands at the elite level across international football and top domestic leagues. This suggests that elite players, regardless of the competitive context, are subjected to similar physical demands, emphasizing the high level of preparation and fitness required to perform at the elite level. 

\subsubsection{Predictive Variables}
XG-Boost Model: Geurkink et al. (2021) \cite{Geurkink2021} achieved a predictive accuracy of 89.6\% for forecasting match outcomes in the Belgian Professional Soccer division. This study pinpointed "shots on target from the attacking penalty box" and "frequency of high-intensity runs" as the most influential variables for predicting the results of soccer matches within this specific league.

\subsubsection{Contextual Variables}
World Cup Success Factors: Lepschy et al. (2021) \cite{Lepschy2021} analyzed matches from the FIFA 2018 World Cup in Russia and FIFA 2014 World Cup in Brazil. The most significant insight from Lepschy et al.'s study is the critical role of defensive actions in securing match victories. Defensive errors, shots from counterattacks, and successful tackles emerged as significant indicators of success, emphasizing the strategic importance of robust defensive play and the effective utilization of counterattacks. Contrary to popular belief, possession metrics and the overall market value of a team were not significantly correlated with success. It's important to recognize some of the limitations of this study, such as being focused solely on international football, which makes a big contextual difference in terms of player fatigue, psychological pressures, new play styles, and so on.

